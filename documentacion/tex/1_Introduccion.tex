\capitulo{1}{Introducción}

Desde hace miles de años, el ser humano ha contemplado las estrellas en el cielo nocturno, sintiéndose fascinado por esa multitud de puntos luminosos. Primero construyó leyendas a su alrededor, mitos con los que intentaba comprender su realidad y dar significado a su mundo, pero pronto observó que algunas de esas estrellas no estaban quietas en la bóveda celeste, sino que se movían, realizando complejos círculos que trataron de entender, estudiar y predecir. Sin saber aún lo que eran, estaban observando otros planetas. 

La observación del cielo y de los cuerpos que lo habitan ha sido una actividad continuada durante toda la historia del ser humano. Con el paso de los siglos y el desarrollo de nuevas ideas y tecnologías, nuestro conocimiento del cosmos no ha parado de crecer, primero en nuestro entorno \textit{cercano}, para ir, en los últimas décadas, adentrándose en regiones cada vez más alejadas de nuestro sistema solar. Estrellas y planetas han perdido su carácter místico, dando lugar a teorías sobre su formación, su ciclo vital y su muerte.

Aún así, la mayor parte del universo permanece desconocido. Sólo en nuestra galaxia, la Vía Láctea, se estima que existen entre 100.000 y 400.000 millones de estrellas y, según los últimos datos, el número de galaxias estimado en el universo observable es de unos dos billones. Algunas estimaciones sobre el número de planetas, según consideremos su número medio por estrella, lo sitúan en torno a $10^{25}$. Y, sin embargo, no fue hasta 1995, con el descubrimiento de Dimidium, que tuvimos constancia del primer exoplaneta.

Aunque el número de exoplanetas descubiertos ha ido creciendo con los años, la pregunta es obvia, ¿cómo es, entonces, que habiendo tantos , conocemos tan pocos? Y, más importante aún, ¿podemos hacer algo para detectar más exoplanetas y hacerlo de forma más rápida? En este trabajo estudiaremos técnicas que intentarán responder dicha pregunta.    

\section{Kepler y la detección de exoplanetas}

Kepler es un telescopio espacial lanzado por la NASA el 7 de marzo de 2009. Nombrado así en honor al astrónomo alemán Johannes Kepler y colocado en órbita heliocéntrica, el objetivo de la misión era buscar planetas extra solares, especialmente aquellos de un tamaño similar a la Tierra, situados en la zona de habitabilidad de su estrella.

La detección de exoplanetas de forma directa, esto es, observándolos directamente mediante un telescopio, es una tarea muy complicada, siendo muy pocos los descubiertos de esta forma. Ello se debe principalmente a que los planetas no emiten luz propia, sino que simplemente reflejan parte de la luz de sus estrellas. Siendo, además, muy pequeños en comparación con su estrella, es fácil que su brillo quede eclipsado por el de su estrella madre. Así pues, los planetas detectados de esta forma suelen tener dos características comunes: son gigantes gaseosos muy alejados de su estrella. 

Sin embargo, es posible detectar exoplanetas de formas indirectas. Una de ellas, la usada por el telescopio Kepler, es el conocido método del tránsito. En términos astronómicos, un tránsito ocurre cada vez que un objeto pasa por delante de otro mayor, bloqueando su visión. El ejemplo más directo es un eclipse solar, durante el cual la Luna se coloca entre la Tierra y el Sol, bloqueando de forma total o parcial la visión de este. De la misma forma, si estuviésemos observando cualquier estrella y un planeta pasase por delante de ella, notaríamos una disminución en la intensidad de su luz.

Dotado de un sensible fotómetro y con un campo de visión fijo, Kepler fue apuntando hacia las constelaciones del Cisne, Lira y Dragón para captar de forma simultánea la luz emitida por unas 150.000 estrellas. La duración inicial de la misión estaba prevista en tres años y medio, pero el ruido generado en los datos estaba haciendo mayor de lo esperado, lo que hacía necesario un mayor tiempo para completar sus objetivos. El plazo de la misión fue extendido hasta 2016, pero el infortunio hizo que dos de los giroscopios se estropearan, uno a mediados del 2012 y otro en mayo del 2013. Con solo otros dos giroscopios operativos, la misión original tuvo que ser abandona. En su lugar, tras escuchar diferentes alternativas por parte de la comunidad científica, la NASA aprobó la nueva misión de Kepler, denominada K2, Segunda Luz.    

\imagenflotante{transito_exoplaneta.jpg}{Disminución del flujo de luz durante el transito planetario\cite{TransitoExoplaneta}}

El 30 de octubre del 2018, Kepler agotó totalmente su combustible y fue apagado por la NASA. Durante sus nueve años de servicio, Kepler observó 530.506 estrellas y descubrió 2.662 exoplanetas, aproximadamente un 70\% de todos los que conocemos. Sus datos nos permitieron estimar que en sólo en la Vía Láctea existen por lo menos otros 17.000 millones de exoplanetas de tamaño similar al nuestro. Muchos de estos datos aún siguen estudiándose; enterrados en esos datos hay otros 2900 planetas aún sin confirmar. Pero la búsqueda no termina aquí: el 18 de abril del mismo año, la NASA lanzaba el telescopio TESS para continuar la detección de nuevos mundos usando, igualmente, el método del tránsito. Y en 2021 está previsto el lanzamiento del telescopio James Webb con la tarea de examinar los hallazgos más prometedores de Kepler y TESS. ¿Cuantos nuevos planetas nos ayudaran a descubrir?

\section{Machine learning y la busqueda de exoplanetas}

Dada la evidente cantidad de información obtenida por el telescopio Kepler así como su complejidad, es obvia la necesidad de automatizar el proceso de la búsqueda de exoplanetas. Para ello se han utilizado diferentes algoritmos como VARTOOLS\cite{2016A&C....17....1H} o Transit Least Squares (TLS)\cite{2019A&A...623A..39H}, ambos basados en estudiar los periódicos picos de caída de la intensidad de la luz\cite{Minsky-1969}.

Dado que la heterogeneidad de los datos hace bastante complejo su estudio con un enfoque algorítmico tradicional, ha surgido otro enfoque distinto, basado en técnicas de aprendizaje automático. Se han propuesto diversos modelos para encontrar soluciones adecuadas, caracterizadas principalmente por el tipo de arquitectura de red utilizada. Así, podemos encontrar soluciones basadas en arboles de decisión, perceptrones multicapa, redes recurrentes, convolucionales o varias de ellas mezcladas. Pero también se han propuesto alternativas que buscan estudiar no las variaciones del flujo de luz, sino su frecuencia. En cualquier caso, esto no quita que estas aproximaciones no adolezcan, también, de otras tantas dificultades. En entre ellas, resaltan dos: la gran cantidad de ruido en los datos y la escasez de ejemplos de exoplanetas confirmados.

En este trabajo intentaremos encontrar soluciones mediante el aprendizaje automático. Para ello construiremos modelos con diferentes arquitecturas y realizaremos análisis tanto del flujo de luz como de su frecuencia. Trabajaremos, además, con técnicas para reducir el nivel de ruido de los datos y estudiaremos si es posible mejorar nuestros modelos mediante la generación automática de ejemplos positivos adicionales. 