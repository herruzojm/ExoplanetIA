\capitulo{2}{Objetivos del proyecto}

\begin{itemize}
    \item Investigar las técnicas y soluciones existentes de aprendizaje automático para la detección de exoplanetas mediante el análisis del flujo de luz.
    \item Comparar diversas librerías de aprendizaje automático para, en base a sus características, elegír la más adecuada para implementar modelos con diferentes arquitecturas e hiperparámetros.
    \item Estudiar los datos disponibles, las diferentes formas de procesarlos y como este procesado puede ayudarnos a obtener mejores resultados.
    \item Diseñar, implementar y comparar diferentes soluciones, seleccionando el modelo que presente mejores resultados.
    \item Comparar nuestro modelo con otros presentados en la plataforma de Kaggle.
    \item Desarrollar una pequeña aplicación web que permita ejecutar el modelo seleccionado. La aplicación debe permitir cargar un fichero con datos de flujos de luz, analizarlos y presentar los resultados.
    \item Utilizar un sistema de control de versiones para gestionar los cambios en el código.
    \item Utilizar una metodología ágil para el desarrollo y la planificación del proyecto.
\end{itemize}
