\capitulo{2}{Objetivos del proyecto}\label{sec:objetivos-del-proyecto}

Este proyecto se encuentra especialmente hacia la investigación y el desarrollo de modelos de aprendizaje automático capaces de resolver un problema real, como es detectar, mediante el método del tránsito, la presencia de exoplanetas orbitando una estrella. Además, una vez se haya completado la investigación y se disponga de varios modelos, se probarán sus capacidades y se seleccionarán algunos para sea posible ejecutarlos en una aplicación web diseñada a tal efecto. 

En base a ello, podemos dividir el proyecto en dos fases, investigación de modelos y desarrollo web. En la parte de investigación nos marcamos los siguientes objetivos:

\begin{itemize}
    \item Investigar las técnicas y soluciones existentes de aprendizaje automático para la detección de exoplanetas mediante el análisis del flujo de luz.
    \item Comparar diversas librerías de aprendizaje automático para, en base a sus características, elegir la más adecuada para implementar modelos con diferentes arquitecturas e hiperparámetros.
    \item Estudiar los datos disponibles, las diferentes formas de procesarlos y cómo este procesado puede ayudarnos a obtener mejores resultados.
    \item Diseñar, implementar y comparar diferentes soluciones, seleccionando el modelo que presente mejores resultados.
    \item Comparar nuestro modelo con otros presentados en la plataforma de Kaggle.    
    \item Utilizar un sistema de control de versiones para gestionar los cambios en el código.
    \item Utilizar una metodología ágil para el desarrollo y la planificación del proyecto.
\end{itemize}

Para el desarrollo web tenemos los siguientes objetivos:

\begin{itemize}
    \item Desarrollar una pequeña aplicación web que permita ejecutar el modelo seleccionado. La aplicación debe permitir cargar un fichero con datos de flujos de luz, analizarlos y presentar los resultados de la detección de exoplanetas.
    \item Aplicar un flujo de desarrollo del software para asegurar la calidad y el despliegue continuo.
    \item Facilitar la portabilidad de la aplicación mediante su distribución en contenedores.
    \item Utilizar un sistema de control de versiones para gestionar los cambios en el código.
    \item Utilizar una metodología ágil para el desarrollo y la planificación del proyecto.    
\end{itemize}