\capitulo{3}{Conceptos teóricos}

En esta sección se presentan los conceptos teóricos relacionados con este Trabajo Fin de Grado que  sirven para facilitar la comprensión. TODO




%\section{Tablas}

%Igualmente se pueden usar los comandos específicos de \LaTeX o bien usar alguno de los comandos de la plantilla.

%\tablaSmall{Herramientas y tecnologías utilizadas en cada parte del proyecto}{l c c c c}{herramientasportipodeuso}
%{ \multicolumn{1}{l}{Herramientas} & App AngularJS & API REST & BD & Memoria \\}{ 
%HTML5 & X & & &\\
%CSS3 & X & & &\\
%BOOTSTRAP & X & & &\\
%JavaScript & X & & &\\
%AngularJS & X & & &\\
%Bower & X & & &\\
%PHP & & X & &\\
%Karma + Jasmine & X & & &\\
%Slim framework & & X & &\\
} 

\section{Proceso experimental para la generación de modelos}
TODO interesante  comentar el marco global del proceso de creación de modelos para facilitar la lectura. Puedes ser interesantes algo similar al proceso CRISP-DM.


\section{Repositorios de datos Kaggle-Kepler}

TODO Kaggle \footnote{ \url{https://www.kaggle.com/keplersmachines/kepler-labelled-time-series-data}}.

Descripción del conjunto de datos y de los conjuntos de datos disponibles


\section{Perceptrón Multicapa}\label{perceptron-multicapa}

El perceptrón multicapa, o red multicapa con propagación hacia delante,
es el modelo de aprendizaje profundo por excelencia. Es una
generalización del perceptrón simple que surgió debido a la incapacidad
de estos para dar solución a problemas no lineales \cite{Minsky-1969}.

El objetivo de estas redes es aproximar alguna función \emph{f}. Por
ejemplo, para un clasificador como es nuestro caso,
\$\texttt{y = f(x)}\$ mapea un input, \emph{x} a una categoría \emph{y}.
La red define un mapeado \$\texttt{y = f(x, \textbackslash{}theta)}\$ y
aprende los valores de los parametros \$\texttt{\textbackslash{}theta}\$
que resultan en la mejor aproximación a la función \cite{Goodfellow-et-al-2016}.

Su arquitectura es simple, consistiendo en una capa de entrada,
encargada de recibir las señales del exterior y propagarlas a las
neuronas de la siguiente capa, una o más capas ocultas, que procesan la
información, aplicando una función de activación a los datos recibidos
de la capa previa, y una capa de salida, que comunica al exterior la
respuesta de la red.

\imagen{perceptron-multicapa.png}{Perceptrón multicapa}


\subsection{Consideraciones de diseño}\label{consideraciones-de-diseno}

A la hora de diseñar la arquitectura de un perceptrón multicapa hay
varios elementos que podemos alterar para tratar de lograr una mejor
solución.

\subsection{Número de neuronas}\label{numero-de-neuronas}

En algunos casos, especialmente en la capa de entrada y de salida, el
número de neuronas viene definido por el problema a resolver. En las
capas ocultas, este número puede variar ampliamente. Hay que considerar
que un número elevado de neuronas puede provocar que estas memorizen los
datos de entrada, proceso conocido como \emph{overfitting}. En este
caso, nuestra red proporcionaría buenos resulados durante con los datos
entrenamiento, pero sería incapaz de generalizar y los resultados serian
pobres cuando se enfrentase a datos nuevos.

Por otro lado, un número escasos de neuronas puede provocar el efecto
contrario, que nuestra red no disponga de la capacidad necesaria para
generalizar correctamente. En este caso, conocido como
\emph{underfitting}, la red presenta pobres resultados, tanto en el
entrenamiento como en los tests.

En nuestro caso, el número de neuronas en la capa de entrada viene
determinado, a priori, por el número de características de nuestro
dataset, esto es, 3197. Respecto a la capa de salida, dependerá de la
función de activación que se vaya usar; en el problema que tratamos de
resolver, clasificando los datos en dos categorías, hay o no hay
exoplaneta, usaremos dos neuronas.

\subsection{Número de capas y
	conexiones}\label{numero-de-capas-y-conexiones}

De forma similar al número de neuronas, el número de capas ocultas puede
variar ampliamente, contribuyendo especialmente al problema de
\emph{overfitting} comentado anteriormente. Además, de acuerdo al
teorema de aproximación universal, cuando se usan funciones de
activación no lineales, una sola capa oculta es suficiente para
representar cualquier función continua en un rango dado, aunque esta
capa puede ser demasiado grande y fallar en aprender y generalizar
correctamente \cite{Goodfellow-et-al-2016}, por lo que es conveniente probar varias
aproximaciones. Dado que nuestro problema es, además, no continuo, no
debemos ceñirnos a usar una sola capa oculta.

Es también importante como se encuentran conectadas las capas. El modelo
es más frecuente es el de capa totalmente conectada, donde cada neurona
esta conectada a cada una de las neuronas de la siguiente capa. Hay, sin
embargo, otras opciones donde, la más frecuente de ellas, consiste en
que la salida de algunas o todas las neuronas de una capa se conectan
con la entrada de neuronas de una capa no inmediatamente posterior,
haciendo que su valor tenga más peso en el resultado final de la red.

Otro opción respecto a las conexiones entre las capas es usar la técnica
conocida como \emph{dropout}. Esta consiste en asignar, durante el
proceso de entrenamiento, el peso de determinadas neuronas,
seleccionadas de forma aleatoria, a cero, excluyendo de facto su
aportación al resultado final de la red. El objetivo de la técnica es
reducir el \emph{overfitting}, ya que hace que la red sea menos
dependiente del peso especifico de determinadas neuronas \cite{JMLR:v15:srivastava14a}.

\subsection{Funciones de activación}\label{funciones-de-activacion}

Vamos a examinar brevemente las funciones de activación más frecuentes
que podríamos usar en nuestro modelo.

La función sigmoide fué de las primeras en usarse de forma masiva. La
función está acotada entre {[}0, 1{]} y suele usarse en la última capa
para representar probabilidades en clasificadores binarios. También es
habitual usarla en las capas ocultas de los perceptrones multicapa. Sin
embargo, adolece de algunos problemas, quizá el mayor de ellos sea que
satura y mata el gradiente, provocando una lenta convergencia.

La tangente hiperbólica es muy similar a la sigmoide, estando igualmente
acotada, aunque en un rango mayor, {[}-1, 1{]}, lo que la hace adecuada
para problemas en los que hay que decidir entre dos opciones. A
diferencia de la sigmoide, esta centrada en el 0.

Es importante resaltar que la función sigmoide y la tangente hiperbólica
se encuentran relacionadas, tal que \$tanh(x) = 2 * sigmoid(2x) -1\$ por
loq que existe poca diferencia a la hora de usar una u otra.

Tenemos también la función relu, función lineal rectificada por sus
siglas en inglés. Esta función no está acotada y deja los valores
positivos sin alterar pero transforma los negativos en cero. Tiene un
buen desempeño en redes convolucionales a la hora de tratar con
imagenes, pero también es la opción por defecto en los perceptrones
multicapa. Esto se debe principalmente a dos factores: es poco probable
que mate el gradiente y genera redes escasas, esto es, redes con
neuronas muertas que no se activan, lo que hace la red más eficiente.
Además, su facilidad de cálculo respecto a otras funciones, acorta el
tiempo de entrenamiento de la re. Presenta, sin embargo, un importante
problema, y es que puede matar a demasiadas neuronas. Para solucionarlo,
existe una variante, denominada leaky relu, que, en lugar de anular los
valores negativos, los multiplica por un coeficiente para devolver un
valor negativo.

Finalmente, hablamos de la funcion softmax, que transforma un vector de
entrada en un vector de probabilidades cuyo sumatorio es uno. Es usada
frecuentemente en la capa de salida de la red cuando se trata de
resolver un problema de clasificación.

De cara al diseño de nuestro modelo, la elección evidente para la capa
de salida es usar una función softmax, aunque también se realizarán
pruebas usando la función sigmoide para ver si presenta un mejor
resultado.

Respecto a las capas ocultas, usaremos principalmente la función relu y,
de forma similar a con la capa de salida, haremos pruebas con la
sigmoide.

\subsection{Algoritmo de optimización}\label{algoritmo-de-optimizacion}

El algoritmo de optimización es el encargado de actualizar los pesos de
nuestra red para minimizar la perdida. Pytorch ya tiene implementados
varios de estos algoritmos, por lo que solo queda decidir cual usar.

El más conocido y uno de los primeros en desarrollarse es el descenso
del gradiente. Pytorch implementa el descenso del gradiente estocástico
(SGD), que actualiza los pesos tras procesar cada ejemplo, en lugar de
hacerlo tras procesar todo el dataset. Este algoritmo presenta algunas
dificultades, como elegir la tasa de aprendizaje adecuada y que ese
valor se aplique a todos los pesos por igual, así como oscilaciones que
dificultan la convergencia en el punto mínimo. Para intentar corregir
este último problema, el algoritmo puede configurarse para usar momento,
que ayuda a suavizar las oscilaciones añadiendo una fracción de los
pasos previos al paso actual.

Usaremos este optimizador como linea base de trabajo con diferentes
valores para la tasa de aprendizaje tanto con como sin momento.

Otro de los algoritmos más usados es Adam, acrónimo en inglés de
estimación adaptativa del momento, que será el otro algoritmo que
usaremos.

A diferencia de SGD, Adam calcula tasas de aprendizaje distintas para
los parámetros e incorpora momento. Adam es computacionalmente
eficiente, tiene pocos requisitos de memoria y facilita la convergencia.
Además, al actualizar los parámetros con diferentes tasas de
aprendizaje, es menos sensible a una elección no óptima de la tasa de
aprendizaje inicial.

\section{Medidas de desempeño del modelo}\label{metricas}

El objetivo de nuestro modelo es obtener un clasificador binario que nos
indique la probabilidad de que una entrada de datos pertenezca a una de
nuestras dos clases: exoplaneta y no exoplaneta.

Con este objetivo en mente, la literatura existente nos indica que la
solución óptima suele ser aplicar una función de activación softmax para
la capa de salida de la red. Esta función, también llamada función
exponencial normalizada, es una forma de regresión logística que
normaliza un valor de entrada en un vector de salida que sigue una
distribución de probabilidad cuya suma total es 1. Así pues, el valor de
salida de la neurona k-ésima vendrá dado por la función:

\begin{math}
s(x_{i})=\frac{e^{x_{i}}}{∑_{j=1}^{n}e^{x_j}}
\end{math}

En cualquier caso, estudiaremos otras opciones, como puede ser el caso
de la función sigmoide, que nos devuelve un valor en el rango {[}0,
1{]}, el cual puede ser interpretrado como probabilidad, en nuestro
caso, de que sea una estrella con exoplaneta.

\subsection{Desbalanceo del dataset}\label{desbalanceo-del-dataset}

Analizando nuestro conjunto de datos, observamos que este se encuentra
muy desbalanceado: los casos negativos (no es un exoplaneta) superan
ampliamente en número a los casos positivos (si es un exoplaneta).

Esto supone un problema para el aprendizaje de la red ya que, ante
cualquier entrada, esta puede \emph{``aprender''} a responder siempre
que no es un exoplaneta, acertando en la amplia mayoría de los casos.

Para solventar este problema, siguiendo a \cite{Viloria-2006}, vamos a
definir nuestra función de evaluación, con la que juzgaremos el
aprendizaje real de nuestra red y su capacidad de predecir el resultado
correcto frente a nuevas entradas de datos.

\begin{math}
f = Acierto * (\alpha * Sen + \beta * Esp)
\end{math}

donde \$\texttt{Acierto}\$ representa el ratio de respuestas correctas,
\$\texttt{Sen}\$ es la sensibilidad (casos catalagocados como positivos
que son realmente positivos), \$\texttt{Esp}\$ es la especificidad
(casos negativos correctamente calificados como no exoplanetas) y
\$\texttt{\textbackslash{}alpha}\$ y \$\texttt{\textbackslash{}beta}\$
son dos pesos usados para alterar la importancia de la sensibilidad y la
especificidad. Comenzaremos con un valor neutro de 0.5 para cada uno,
pero estudiaremos si su ajuste permite obtener un mejor modelo.

Definimos a continuación los ratios de \$\texttt{Acierto}\$,
\$\texttt{Sen}\$ y \$\texttt{Esp}\$, donde \$\texttt{VP}\$ es el número
de verdaderos positivos, \$\texttt{VN}\$ los verdaderos negativos,
\$\texttt{FP}\$ los falsos positivos y \$\texttt{FN}\$ los falsos
negativos.

\begin{math}
Acierto = \frac{VP + VN}{VP + VN + FP + FN}
\end{math}

\begin{math}
Sen = \frac{VP}{VP + FN}
\end{math}

\begin{math}
Esp = \frac{VN}{VN + FP}
\end{math}



\section{Bibliotecas de machine learning}\label{bibliotecas-de-machine-learning}

De cara a implementar nuestro modelo de red neuronal debemos decidir que
lenguajes y bibliotecas vamos a usar. A día de hoy, la mayor parte de los
frameworks y bibliotecas que facilitan el desarrollo de redes neuronales
funcionan en entornos Python, que puede ser considerado el lenguaje de
facto de la industria, aunque existen otras alternativas en lenguajes
como R, Mathlab, y en frameworks como Neuroph para Java o Mathematica.

En nuestro caso, vamos a considerar una comparativa de tres bibliotecas de Python:

\begin{itemize}
	\itemsep1pt\parskip0pt\parsep0pt
	\item
	Tensorflow es una biblioteca de código abierta desarrollada por Google
	para uso interno, tanto en investigación como en producción, que
	posteriormente fue lanzada al público.\\
	\item
	Keras es una API de alto nivel capaz de ejecutarse sobre otros
	lenguajes o bibliotecas, como Tensorflow, R o Theano, diseñada con el
	foco en la facilidad de uso.\\
	\item
	PyTorch es una biblioteca de código abierta desarrollada principalmente
	por Facebook que también presenta una interfaz para C++.
\end{itemize}

Vamos a examinar diferentes parámetros para ver que nos aporta cada una
de ellas.

\subsection{Velocidad}\label{velocidad}

Los estudios muestran que no hay una diferencia significativa de
velocidad entre Tensorflow y PyTorch. Este no es el caso con Keras, que
presenta un rendimiento claramente inferior.

\subsection{Nivel del API}\label{nivel-del-api}

Como se ha comentado, Keras es una API de alto nivel, capaz de correr
sobre otras bibliotecas, como Tensorflow o Theano, proporcionando una
interfaz común que facilita el desarrollo rápido de proyectos.

Tensorflow proporciona APIs tanto de alto como de bajo nivel, lo que le
dota una gran flexibilidad.

Finalmente, PyTorch proporciona sólamente una API de nivel, enfocada en
el trabajo directo con matrices.

\subsection{Arquitectura}\label{arquitectura}

Keras presenta una arquitectura simple y fácil de comprender, mientras
que tanto Tensorflow como PyTorch presentan arquitecturas más complejas
y un código con mayor verbosidad.

La API de PyTorch se encuentra mejor diseñada mientras que la de
Tensorflow es un tanto confusa y ha recibido numerosos cambios
importantes en cada versión, lo que dificulta mantener un código estable
y estar actualizado.

\subsection{Debuggin}\label{debuggin}

Depurar código en Tensorflow es relativamente complejo y no muy
intuitivo. En Keras no es un proceso habitual, dado el alto nivel de sus
componentes, lo que tampoco facilita la depuración en caso de algún
problema, ya que la mayor parte del código se encuentra en la biblioteca.
Sin embargo, PyTorch si ofrece buenas opciones para la depuración, muy
similares a las encontradas en IDEs para lenguajes conocidos, como
Eclipse o Visual Studio.

\subsection{Dataset}\label{dataset}

Los problemas de velocidad de Keras no lo hacen adecuado para trabajar
con grandes datasets. No es el caso de Tensorflow o PyTorch, que no
tienen este problema de rendimiento.

\subsection{Documentación y comunidad}\label{documentacion-y-comunidad}

Tanto en PyTorch como en Tensorflow se nota el efecto de tener detras a
dos grandes empresas tecnológicas. En ambos casos, existen númerosos
recursos gratuitos con los que aprender así como una importante
comunidad de usuarios que las respaldan y ofrecen su ayuda. Tensorflow
tiene más tiempo de desarrollo y su base de usuarios es mayor pero desde
el 2018 la popularidad de PyTorch está en constante aumento,
especialmente en el ambito académico.

Keras contrasta respecto a las otros dos con una más reducida comunidad
y menor documentación.

\subsection{Puesta en producción}\label{puesta-en-produccion}

A la hora de poner en producción un modelo previamente entrenado, Keras
no dispone de ninguna utilidad en si misma, haciendo uso de las
caracteristicas de Tensorflow. Este permite servir los modelos en un
servidor web mediante una API REST o en dispositivos móviles.

PyTorch se apoya en otras bibliotecas para poder exponer sus modelos via
web, permitiendo también otras opciones interesantes, como la interfaz
con C++, lo que permite convertir los modelos en ejecutables fácilmente.

\subsection{Resumen de la comparación }\label{decision-final}

Tras analizar las características de las tres bibliotecas, vemos como se
adaptan a nuestras necesidades.

Keras es una buena opción para probar y generar modelos de forma rápida,
pero no nos permite profundizar en el aprendizaje y compresión de las
redes neuronales, ya que la mayor parte del trabajo de nivel es
gestionado de forma interna por la biblioteca. Unido a la dificultad de
depurar el código y a su peor rendimiento, hace que optemos por no
utilizarla.

La decisión entre Tensorflow y PyTorch es más dificil de realizar, ya
que ambos aportan características similares: la posibilidad de trabajar
con las redes a bajo nivel para poder estudiarlas en detalle, buen
rendimiento y variados recursos para aprender, ya sea en forma de
tutoriales o de comunidad de usuarios para resolver dudas. Sin embargo,
hay dos detalles marcan la diferencia y nos hacen decantarnos por
PyTorch: por un lado, la facilidad de depuración del código y, por otro,
la posibilidad de generar ejecutables.

Así pues, la biblioteca que usaremos para la realización del proyecto será
\textbf{PyTorch}.
