\capitulo{4}{Técnicas y herramientas}

Se exponen a continuación las herramientas usadas durante el desarrollo del proyecto.

\section{Python}\label{sec:bibliotecas-de}
Python es un lenguaje de programación interpretado de alto nivel. Posee licencia de código abierto y, en los últimos años, se ha convertido en el estandar de facto para los proyectos de machine learning.

\section{Jupyter Notebook}
IDE interactivo de código abierto basado en web. Permite crear y compartir documentos que contienen tanto código como texto markdown.  

\section{Bibliotecas de Pyhton}

\subsection{Torch}
Biblioteca de código abierto para aprendizaje automático. Se puede encontrar más información sobre sus características y su comparación con otras bibliotecas similares en la \autoref{sec:bibliotecas-de-machine-learning} \nameref{sec:bibliotecas-de-machine-learning}\
 
\subsection{NumPy}
Biblioteca que agrega mayor soporte para vectores y matrices, constituyendo una biblioteca de funciones matemáticas de alto nivel para operar con esos vectores o matrices.

\subsection{Pandas}
Extensión de NumPy para manipulación y análisis de datos. Ofrece estructuras de datos y operaciones para manipular tablas numéricas y series de tiempo.

\subsection{SciPy}
Basado en NumPy, expande esta biblioteca con herramientas y algoritmos para matemáticas, ciencias e ingeniería. 

\subsection{imbalanced-learn}
Ofrece técnicas y algoritmos de \textit{under-sampling} y \textit{over-sampling} comúnmente utilizados en conjuntos de datos que muestran un fuerte desequilibrio entre clases.

\subsection{Matplotlib}
Matplotlib es una biblioteca para crear visualizaciones estáticas, animadas e interactivas en Python.

\subsection{Os}
Proporciona una interfaz para utilizar los comandos del sistema operativo.

\subsection{Time}
Aporta funcionalidades para trabajar con objetos de fechas y horas.

\subsection{Repackage}
Biblioteca para invocar paquetes no registrados y acceder a ellos con rutas relativas.

\subsection{HTML}
Funcionalidades para trabajar con documentos HTML.

\subsection{Flask}
Framework minimalista para crear aplicaciones web de forma rápida y sencilla.

\subsection{Werkzeug}
Biblioteca que proporciona diferentes utilidades relativas al desarrollo web.

\subsection{Wtforms}
Añade representación y validación de formularios flexible para el desarrollo web con Python.

\subsection{Flask-wtf}
Integra la biblioteca Wtforms con Flask.

\subsection{Base64}
Proporciona funcionalidades para codificar y decodificar datos binarios.

\section{Git}
Sistema de control de versiones distribuido de código abierto.

\section{Gitlab}
Servicio web de control de versiones y desarrollo de software colaborativo basado en Git. Es una suite completa que permite gestionar, administrar, crear y conectar los repositorios con diferentes aplicaciones y hacer todo tipo de integraciones con ellas, ofreciendo una plataforma en la cual se puede realizar todas las etapas del ciclo de desarrollo del software.

\section{LaTeX}
Sistema de composición de textos, orientado a la creación de documentos escritos que presenten una alta calidad tipográfica. Es usado de forma especialmente intensa en la generación de artículos y libros científicos que incluyen, entre otros elementos, expresiones matemáticas.

\subsection{MiKTeK}
Distribución de LaTex para sistemas Windows.

\subsection{TexMaker}
Editor de código abierto de LaTex. Integra variedad de herramientas para desarrollar documentos.