\apendice{Especificación de Requisitos}

\section{Introducción}

A continuación, procedemos a abordar los requisitos del proyecto.

\section{Objetivos generales}

Este proyecto de investigación nace con los siguientes objetivos:

\begin{itemize}
    \item Investigar las técnicas y soluciones existentes de aprendizaje automático para la detección de exoplanetas mediante el análisis del flujo de luz.    
    \item Estudiar los datos disponibles, las diferentes formas de procesarlos y como este procesado puede ayudarnos a obtener mejores resultados.
    \item Diseñar, implementar y comparar diferentes soluciones, seleccionando el modelo que presente mejores resultados.
    \item Disponer de software que facilite el uso de diferentes técnicas y el desarrollo de modelos alternativos a los aquí presentados.
\end{itemize}

Además, durante el desarrollo del proyecto se considera añadir algunos nuevos requisitos de importancia secundaria:

\begin{itemize}    
    \item La creación de una interfaz gráfica que permita ejecutar el modelo para comprobar los datos contenidos en un archivo.
    \item Desarrollar un pipeline de despliegue continuo que permita desplegar de forma automatica cualquier cambio en nuestro modelo o en la interfaz grafica.
\end{itemize}

\section{Catalogo de requisitos}

Se muestran a continuación las características que debe cumplir nuestro proyecto, clasificándolas en dos grupos, requisitos funcionales y no funcionales. Mientras que los primeros definen un comportamiento exacto del software, es decir, \textit{que} hace, los segundos hacen referencia a sus propiedades, es decir, \textit{como} las hace.

Definimos dos actores:

\begin{itemize}
    \item \textbf{Investigador}: persona que realiza la investigación y utiliza el software para generar y entrenar modelos.
    \item \textbf{Usuario}: persona que usa el software para determinar si un conjunto de estrellas tienen o no exoplanetas orbitando en torno a ellas.
\end{itemize}

\subsection{Requisitos funcionales}
\begin{itemize}

    \item \textbf{RF-1} El investigador debe poder dividir un conjunto de datos en sets de entrenamiento y validación.
        \begin{itemize}
            \item \textbf{RF-1.1} Se debe indicar que porcentaje respecto al total de instancias se incluiran en el set de validación.
            \item \textbf{RF-1.2} La inclusión de una instancia concreta en un set u otro debe ser aleatoria pero se podrá indicar una semilla para que, dado un set de entrada concreto y un porcentaje determinado, la división produzca los mismos set de entrenamiento y validación.            
        \end{itemize}
    
    \item \textbf{RF-2} El investigador debe poder aplicar técnicas y algoritmos que prepararen los datos.
        \begin{itemize}
            \item \textbf{RF-2.1} Debe poder normalizar los datos.
            \item \textbf{RF-2.2} Debe poder aplicar un filtro gaussiano.
            \item \textbf{RF-2.3} Debe poder eliminar los picos anormales de intensidad de luz.
            \item \textbf{RF-2.4} Debe poder transformar el conjunto de datos desde el dominio de la intensidad al dominio de la frecuencia.            
        \end{itemize}

    \item \textbf{RF-3} El investigador debe poder mostrar gráficas de la frecuencia o intensidad de la luz respecto al tiempo.
    
    \item \textbf{RF-4} El investigador debe poder instanciar modelos de redes neuronales.
        \begin{itemize}
            \item \textbf{RF-4.1} Debe poder instanciar modelos de perceptrón multicapa configurables.
            \item \textbf{RF-4.2} Debe poder instanciar modelos de red LSTM configurable.
        \end{itemize}
    
    \item \textbf{RF-5} El investigador debe poder entrenar un modelo.
        \begin{itemize}
            \item \textbf{RF-5.1} El investigador debe poder elegir diferentes algoritmos de optimización o funciones de coste.
            \item \textbf{RF-5.2} Debe mostrarse la evolución del entrenamiento.
            \item \textbf{RF-5.3} Debe guardarse el mejor modelo generado durante el entrenamiento según el nombre y la ruta indicada por el investigador.
            \item \textbf{RF-5.4} El investigador debe poder mostrar gráficamente los resultados del entrenamiento.
        \end{itemize}

    \item \textbf{RF-6} El investigador debe poder probar un modelo guardado.
        \begin{itemize}
            \item \textbf{RF-6.1} El investigador debe poder indicar el nombre y la ruta del modelo a cargar.
        \end{itemize}

    \item \textbf{RF-7} El usuario debe poder analizar un archivo con datos en la aplicación.
        \begin{itemize}
            \item \textbf{RF-7.1} El usuario debe tener información suficiente sobre el formato del archivo a procesar.
            \item \textbf{RF-7.2} El usuario debe obtener que estrellas presentan exoplanetas.
            \item \textbf{RF-7.3} La aplicación debe mostrar las gráficas correspondientes a las estrellas con exoplanetas.
        \end{itemize}

\end{itemize}

\subsection{Requisitos no funcionales}

\begin{itemize}
    \item \textbf{RNF-1} El entrenamiento o el testeo de modelos debe poder llevarse a cabo tanto en CPUs como en GPUs, según el hardware del que el investigador disponga.
    \item \textbf{RNF-2} Debe proveerse una aplicación para que un usuario sin entorno ni conocimientos de programación puede probar alguno de los modelos entrenados.
\end{itemize}

\section{Especificación de requisitos}


\subsection{Diagrama de casos de uso del investigador}

\imagen{casos_uso_investigador.png}{Diagrama de casos de uso del investigador}

\subsection{Diagrama de casos de uso del usuario}

\imagen{casos_uso_usuario.png}{Diagrama de casos de uso del usuario}

\subsection{Especificaciones de los casos de uso}

\begin{table}[]
    \begin{center}    
        \begin{tabular}{| >{\columncolor[gray]{0.7}} p{3cm} | p{9.5cm} | }
        \hline
        Caso de uso      & Entrenar un modelo \\ 
        \hline
        Requisitos       &  RF-5\newline
                            RF-5.1\newline
                            RF-5.2\newline
                            RF-5.3\newline
                            RF-5.4 \\ 
        \hline
        Descripción      & El investigador puede entrenar un nuevo modelo dado un conjunto de datos usando un algoritmo de optimización y una función de coste. \\ 
        \hline
        Precondiciones   &  El investigador debe disponer de un conjunto de datos. \\ 
        \hline
        Acciones         &  1. El investigador carga los datos. \newline 
                            2. Opcionalmente, el investigador puede dividir sus datos en dos conjuntos distintos, entrenamiento y validación. \newline 
                            3. Opcionalmente, el investigador puede preparar los datos usando alguna de las técnicas y herramientas propuestas. \newline 
                            4. Opcionalmente, el investigador visualizar los datos. \newline 
                            5. El investigador elige un tipo de modelo y genera una instancia. \newline 
                            6. El investigador fija los hiperparámetros según deseé. \newline 
                            7. El investigador decide que algoritmo de optimización y función de coste usar. \newline 
                            8. El investigador comienza el entrenamiento. \newline 
                            9. Opcionalmente, cuando el entrenamiento termine, el investigador puede mostrar gráficamente los resultados del mismo.  
                            \\ 
        \hline
        Postcondiciones  &  Se ha generado un modelo y las imagenes con los resultados del entrenamiento.\\ 
        \hline
        Excepciones      &  Formato de datos incorrecto. Algoritmo de optimización invalido. Función de coste invalida.Configuración de la instancia del modelo incorrecta.\\ 
        \hline
        Importancia      &  Alta\\ 
        \hline
        \end{tabular}
    \caption{Caso de uso ''Entrenar un modelo''.}
    \label{tabla:casoUso1}        
    \end{center}
\end{table}

\begin{table}[]
    \begin{center}    
        \begin{tabular}{| >{\columncolor[gray]{0.7}} p{3cm} | p{9.5cm} | }
        \hline
        Caso de uso      & Crear sets de entrenamiento y validación \\ 
        \hline
        Requisitos       &  RF-1\newline
                            RF-1.1\newline
                            RF-1.2  \\ 
        \hline
        Descripción      & A partir de un conjunto de datos, el investigador puede dividirlo en dos de forma aleatoria y segun la proporción deseada. \\ 
        \hline
        Precondiciones   & El investigador debe disponer de un conjunto de datos. \\ 
        \hline
        Acciones         &  1. El investigador selecciona el conjunto de datos. \newline 
                            2. El investigador elige que porcentaje del total ocupará el set de validación \newline
                            3. Opcionalmente, el investigador elige una semilla para la selección aleatoria.   \\ 
        \hline
        Postcondiciones  & El investigador obtiene los datos divididos en dos conjuntos. \\ 
        \hline
        Excepciones      & El conjunto de datos está vacio. La proporción es uno o mayor. \\ 
        \hline
        Importancia      &  Media. \\ 
        \hline
        \end{tabular}
    \caption{Caso de uso ''Crear sets de entrenamiento y validación''.}
    \label{tabla:casoUso1}        
    \end{center}
\end{table}

\begin{table}[]
    \begin{center}    
        \begin{tabular}{| >{\columncolor[gray]{0.7}} p{3cm} | p{9.5cm} | }
        \hline
        Caso de uso      & Preparar los datos \\ 
        \hline
        Requisitos       &  RF-2\newline
                            RF-2.1\newline
                            RF-2.2\newline
                            RF-2.3\newline
                            RF-2.4  \\  
        \hline
        Descripción      & El investigador debe poder aplicar a los datos las funciones usadas durante el entrenamiento: normalización, eliminación de picos de intensidad, filtro gaussiano y transformada de Fourier. \\ 
        \hline
        Precondiciones   & El investigador debe disponer de un conjunto de datos. \\ 
        \hline
        Acciones         &  1. El investigador selecciona el conjunto de datos. \newline 
                            2. El investigador aplica la función correspondiente.   \\  
        \hline
        Postcondiciones  & El conjunto de datos transformado. \\ 
        \hline
        Excepciones      & El conjunto de datos está vacio. \\ 
        \hline
        Importancia      & Media. \\ 
        \hline
        \end{tabular}
    \caption{Caso de uso ''Preparar los datos''.}
    \label{tabla:casoUso1}        
    \end{center}
\end{table}

\begin{table}[]
    \begin{center}    
        \begin{tabular}{| >{\columncolor[gray]{0.7}} p{3cm} | p{9.5cm} | }
        \hline
        Caso de uso      & Mostrar gráficos de los datos \\
        \hline
        Requisitos       &  RF-3 \\  
        \hline
        Descripción      & El investigador puede mostrar la gráfica de cualquiera de las estrellas incluidas en el conjunto de datos. \\ 
        \hline
        Precondiciones   & El investigador debe disponer de un conjunto de datos. \\ 
        \hline
        Acciones         &  1. El investigador selecciona el conjunto de datos. \newline 
                            2. El investigador indica los indices de las estrellas a mostrar. \newline
                            3. El investigador aplica la función.  \\  
        \hline
        Postcondiciones  & Gráfica(s) mostrando los datos correspondientes. \\ 
        \hline
        Excepciones      & El indice de las estrellas seleccionadas no se encuentra en el conjunto de datos. \\ 
        \hline
        Importancia      & Baja. \\ 
        \hline
        \end{tabular}
    \caption{Caso de uso ''Mostrar gráficos de los datos''.}
    \label{tabla:casoUso1}        
    \end{center}
\end{table}

\begin{table}[]
    \begin{center}    
        \begin{tabular}{| >{\columncolor[gray]{0.7}} p{3cm} | p{9.5cm} | }
        \hline
        Caso de uso      & Instanciar modelo de red \\ 
        \hline
        Requisitos       &  RF-4\newline
                            RF-4.1\newline
                            RF-4.2 \\   
        \hline
        Descripción      & El investigador debe poder instanciar modelos de perceptrón o de LSTM. \\ 
        \hline
        Precondiciones   & Ninguna. \\ 
        \hline
        Acciones         &  1. El investigador selecciona el tipo de modelo a instanciar. \newline 
                            2. Opcionalmente, el investigador los parámetros para el modelo. \\ 
        \hline
        Postcondiciones  & Instancia del modelo. \\ 
        \hline
        Excepciones      & Ninguna. \\ 
        \hline
        Importancia      & Alta. \\ 
        \hline
        \end{tabular}
    \caption{Caso de uso ''Instanciar modelo de red''.}
    \label{tabla:casoUso1}        
    \end{center}
\end{table}

\begin{table}[]
    \begin{center}    
        \begin{tabular}{| >{\columncolor[gray]{0.7}} p{3cm} | p{9.5cm} | }
        \hline
        Caso de uso      & Testear un modelo \\ 
        \hline
        Requisitos       &  RF-6\newline
                            RF-6.1\\   
        \hline
        Descripción      & El investigador debe poder hacer predicciones sobre un conjunto de datos con un modelo previamente entrenado. \\ 
        \hline
        Precondiciones   & El investigador debe contar un modelo entrenado guardado en disco. \\ 
        \hline
        Acciones         &  1. El investigador selecciona el conjunto de datos. \newline 
                            2. El investigador instancia un modelo de red acorde al modelo que desee testear. \newline
                            3. El investigador testea el modelo.  \\ 
        \hline
        Postcondiciones  & Matriz de confusión del modelo. \\ 
        \hline
        Excepciones      & El modelo de red instanciado no es el adecuado para el modelo guardado. \\ 
        \hline
        Importancia      & Alta. \\ 
        \hline
        \end{tabular}
    \caption{Caso de uso ''Testear un modelo''.}
    \label{tabla:casoUso1}        
    \end{center}
\end{table}

\begin{table}[]
    \begin{center}    
        \begin{tabular}{| >{\columncolor[gray]{0.7}} p{3cm} | p{9.5cm} | }
        \hline
        Caso de uso      & Analizar un archivo \\ 
        \hline
        Requisitos       &  RF-7\newline
                            RF-7.1\newline
                            RF-7.2\newline
                            RF-7.3 \\   
        \hline
        Descripción      & El usuario debe poder cargar un archivo en la aplicación, obteniendo predicciones sobre que estrellas presentan exoplanetas así como las gráficas de dichas estrellas. \\ 
        \hline
        Precondiciones   & El usuario debe disponer de un archivo con datos. \\ 
        \hline
        Acciones         &  1. El usuario selecciona el archivo de datos. \newline 
                            2. El usuario carga el archivo en la aplicación.  \\ 
        \hline
        Postcondiciones  & Resultado del análisis con las gráficas correspondientes. \\ 
        \hline
        Excepciones      & El archivo no es válido. \\ 
        \hline
        Importancia      & Media. \\ 
        \hline
        \end{tabular}
    \caption{Caso de uso ''Analizar un archivo''.}
    \label{tabla:casoUso1}        
    \end{center}
\end{table}